%!TEX program = xelatex
% 内容来源于《国际交流英语》Chapter 14
\documentclass[aspectratio=169,UTF8,c]{beamer}%比例是16:9,UTF8编码支持,顶部对齐
\usepackage{oucslide}
\usepackage[UTF8]{ctex}
\usepackage{xeCJK}
\usepackage{fontawesome}%加入图标支持
\usepackage{tcolorbox}
\usepackage{verbatim}
\usepackage{caption}
\usepackage{subcaption}
\usepackage{indentfirst}
\setlength{\parindent}{2em}

\newcommand\crule[3][black]{\textcolor{#1}{\rule{#2}{#3}}}
    
%基本信息
\title{Chapter 14}
\subtitle{Talking with Professionals}
\author{Xiaoyang WANG}
\date{\today}
\institute{Linyi University}
\newcommand{\last}{Thanks for Watching!}

\begin{document}

\maketitle

\makeoutline

\section{Purpose of Talking}
\subsection{Purpose of Talking}
\begin{frame}
	\frametitle{14.1 Purpose of Talking}
	Experts and scholars usually have the opportunity to extend communication with others from different fields in an international academic conference. 
	
	However, only through the formal presentations in the conference, it is far from satisfactory to achieve the goal of better communication. Therefore, its necessary to know how to talk with other professionals and experts to promote academic exchanges.
\end{frame}
\begin{frame}
	\frametitle{14.1 Purpose of Talking}
	Talking with other professionals in and out of the meeting will help you know the development in a particular research area, learn from the success and failure of others* study, have a further discussion on some certain topics, and even make friends with scholars or experts coming from different countries. 
	
	Therefore, the purposes of talking with other professionals are to exchange information, understand each other, seek possibilities, and establish friendships.
	
	Although each attendee has his own strong points, the large number of participants and limited time make it impossible for a person to talk with everyone. 
\end{frame}
\begin{frame}
	\frametitle{14.1 Purpose of Talking}
	Generally speaking, you should take the initiative and talk with the following possible partners:
	\begin{itemize}
		\item Question-raisers: They raise relevant questions during discussion sections or have expressed interest in your research.
		\item Interested presenters: They work in the frontier of the field with active thinking and would like to talk with other professionals.
		\item Other experts: Such experts as specialists, professors, journal editors, supervisors, or those who have rich experience are outstanding in a particular field.	
	\end{itemize}

	Would-be talking partners can be selected by considering such factors as profession, information, finance, collaboration, research interest, as well as personality.
\end{frame}

\section{Starting a Talk}

\subsection{Talking with Acquaintances}

\begin{frame}
	\frametitle{14.2 Starting a Talk}
	As the first step of successful talking, to start a talk with others is essential. 
	
	The following will introduce some methods and expressions for talking with different participants.
\end{frame}

\begin{frame}
	\frametitle{14.2.1 Talking with Acquaintances}
	Talking with an acquaintance is simple. The more familiar with each other you are, the simpler opening expressions you will use. Usually this kind of talking does not need roundabout and polite expressions. It tends to go directly to the desired topic. For example:
	\begin{itemize}
		\item How are you doing, Mr. XXX? I just dropped in for a chat.
		\item Hello, George. Are you free now? I'd like to have a talk with you about...
		\item Mr. XXX, shall we go on with the subject we talked about this morning?
		\item Hi! Have you heard about...? That's rather stupid, I'm afraid. What do you think of that?
	\end{itemize}
\end{frame}

\subsection{Talking with Strangers}

\begin{frame}
	\frametitle{Talking with Strangers}
	How to start a talk with strangers is essential. Here are some ways to initiate your talk with strangers.
	
	\emph{\textbf{(1)Self-Introduction}}
	
	To have effective talking, it is important to find the ideal partners. Usually, the identification card or the conference name badge that participants wear may give some clues for you to find the desired partners. Sometimes attendees can introduce themselves to each other. Self-introduction should not be too long, since it only serves as a start. 
\end{frame}
\begin{frame}
	\frametitle{Talking with Strangers}
	The following are some useful expressions in self-introduction:
	\begin{itemize}
		\item Excuse me, are you Mr. XXX? I'm...from...
		\item I believe you are Prof. XXX, aren't you? My name is...
		\item You are Mr. XXX. I think I've seen you somewhere before. I'm...
		\item I don't believe we've met before. Please allow me to introduce myself I'm...
		\item Are you Mr. XXX? My name is... Here is my card. I've heard a great deal about you.
		\item Hello! Aren't you Dr. XXX? I met you at the party last night. Do you remember me? I'm...
	\end{itemize}
\end{frame}
\begin{frame}
	\frametitle{Talking with Strangers}
	\emph{\textbf{(2)A Third Party's Introduction}}
	
	Sometimes you may feel intrusive to directly introduce yourself to a desired talking partner, especially when the desired partner is a celebrity. On such occasion, you may ask someone else who is acquainted with the desired partner to introduce you to the desired would-be talker. For example:
	\begin{itemize}
		\item Mr. XXX, may I ask you to introduce me to your colleague Mr. XXX?
		\item I'd be very much pleased if you could introduce me to Dr. XXX.
		\item I'd like to meet Prof. XXX. Would you kindly do me a favor?
		\item I want to have a talk with Prof. XXX. Would you please be so kind as to introduce me to him?
	\end{itemize}
\end{frame}
\begin{frame}
	\frametitle{Talking with Strangers}
	After being introduced, you should greet the desired would-be partner first and then turn to the desired topic. Although it is the first meeting, talkers had better remember each others name. If the partners name is not heard quite clearly, it is natural to ask again. For example:
	\begin{itemize}
		\item Sony I didn't get the name.
		\item Oh, Graham Nissan, isn't it?
	\end{itemize}
	
	Sometimes if the partners name is long and cannot be pronounced correctly, you may ask for the spelling in order to have a deeper impression. For example:
	\begin{itemize}
		\item Sorry, I can't pronounce your name. How do you spell that?
	\end{itemize}
\end{frame}
\begin{frame}
	\frametitle{Talking with Strangers}
	\emph{\textbf{(3)Balloon D'essai[1]}}
	
	Sometimes, you may want to talk to someone you have not met before, but do not know whether the would-be talker has any interest in talking with you. Then the initiating talker could let out a balloon dessai, i.e. find an excuse to inquire. For example:
	\begin{itemize}
		\item Excuse me, could you please tell me what the speaker's topic is? I missed his presentation just now.
	\end{itemize}
	
	Once hearing the possible answer, you could judge whether your partner would like to continue the talk with you or not. If your partner would, then you may turn to your desired topic from that starter.
\end{frame}
\begin{frame}
	\frametitle{Talking with Strangers}
	\emph{\textbf{(4)Appreciation}}
	
	Most people prefer to hear favorable words. You may start your talk with praising, which usually sounds pleasant, and unexpectedly good effects may be achieved. The following example is the opening of a talk after a presentation:
	\begin{itemize}
		\item Your lecture is very interesting and informative. Indeed, your ideas are the best I've heard on the subject for a long time. What about...?
	\end{itemize}
\end{frame}

\section{Developing a Talk}

\subsection{Disputing Different Opinions}
\begin{frame}
	\frametitle{14.3 Developing a Talk}
	How to develop the topic of talking is a complicated problem that involves many factors. We are going to discuss the matter mainly from the following aspects: 
	
	(a) disputing different opinions; 
	
	(b) shifting to a new topic; 
	
	(c) talking politely and thoughtfully.
\end{frame}

\begin{frame}
	\frametitle{14.3.1 Disputing Different Opinions}
	Disputing different opinions provides an important approach to correction, judgment, mutual inspiration, expanding mind, and learning from others. During a dispute, the two sides not only put forward their reasons and ideas, but also resort to their practical experiences and lessons in research work. This kind of talking and disputing will undoubtedly help push the topic to a higher level.
\end{frame}
\subsection{Shifting to a New Topic}
\begin{frame}
	\frametitle{14.3.2 Shifting to a New Topic}
	During the process of talking, you may feel a certain topic is improper to continue and intend to shift to another one. The reasons for this situation may be as follows:
	
	\begin{itemize}
		\item Your talking partner does not show enough interest in the desired topic.
		\item Your talking partner would like to talk about another topic different from the one you have planned.
		\item The talking partner is less capable in the particular area than you have supposed.
		\item During the talk, both sides have found some new and more significant topics and thus change their original topic.
		\item Some sensitive issues arise during the talk.
		\item There are difficulties caused by language or limited knowledge.
	\end{itemize}
\end{frame}
\begin{frame}
	\frametitle{14.3.2 Shifting to a New Topic}
	Based on the reasons above, there might be two kinds of topic shifting. One is "natural shifting"; which means the two sides have talked over one topic and would like to continue to talk, so they naturally shift to a new topic. The other is 'intended shifting': which means one side intentionally ends the topic and turns to another one. 
	
	If you intend to turn to a new topic, you should be sure to adopt a proper method to keep the friendly air of talking. You can first ask for your partners opinions on something concerned with the new topic. In this way, you can make your partner feel at ease and willing to accept the new topic. 
\end{frame}
\begin{frame}
	\frametitle{14.3.2 Shifting to a New Topic}
	The following are some frequently-used expressions for topic shifting:
	\begin{itemize}
		\item By the way, are you interested in...?
		\item I thought I'd like to talk about...
		\item Er, can we turn our attention to...?
		\item Oh yes. Do you happen to know...?
		\item Well, so much for that. Now, Let's turn to...
		\item Perhaps you'd like to know something about...?
		\item And how did the morning session go? Did you enjoy it?
		\item Well, speaking of... What do you think of the program?
		\item All right.Since you're very interested in..., shall we talk about that?
		\item By the way, I've learned youre specialized in... What about talking a bit about the subject?
	\end{itemize}
\end{frame}

\subsection{Talking Politely and Thoughtfully}

\begin{frame}
	\frametitle{14.3.3 Talking Politely and Thoughtfully}
	Politeness and thoughtfulness show the cultural education and personal characters of participants and are usually helpful for a desired talk to achieve success. Ybu should talk politely and thoughtfully, especially under the following circumstances.
	
	\emph{\textbf{(1)For Personal Opinions}}
	
	After the presentation of ones own ideas, its better to inquire others opinions. For example:
	\begin{itemize}
		\item Thats my opinion. What do you think of it?
		\item Thats what I thought, so what's your opinion of it?
		\item Since I've done that, I hope you'll give me your comments.
		\item Now that you've run over my paper, what is your impression of it?
		\item Now that you've seen what we've done on the subject, I'd like very much to have your views on that.
	\end{itemize}
\end{frame}
\begin{frame}
	\frametitle{14.3.3 Talking Politely and Thoughtfully}
	\emph{\textbf{(2)Being Not Sure of Something}}
	
	An uncertain tone of subjunctive mood is often adopted in saying something that you are not 100\% sure of. For example:
	\begin{itemize}
		\item I wonder if I could...
		\item Personally, I suppose...
		\item I'm afraid it's very unlikely...
		\item I would say that it might...
		\item Perhaps it would...
		\item It's hard to say...
		\item I'm not quite sure of that. It all depends.
	\end{itemize}
\end{frame}

\begin{frame}
	\frametitle{14.3.3 Talking Politely and Thoughtfully}
	\emph{\textbf{(3)Demanding}}
	
	Polite and thoughtful expressions are usually used to see whether the talking partner agrees to do something. For example:
	\begin{itemize}
		\item May I ask you that...?
		\item I wonder if you could... ?
		\item Could you please tell me that...?
		\item Would you mind if I ask...?
	\end{itemize}
	
	\emph{\textbf{(4)Pointing out Mistakes}}
	
	Being polite and thoughtful is especially important when you intend to point out others, inappropriateness or mistakes made during the talk. For example:
	\begin{itemize}
		\item I would say it would be better if...
		\item I thought you might have made a mistake here...
		\item I agree with much of what you said, but one point...
		\item I'm afraid something seems to be wrong in what you said about...
	\end{itemize}
\end{frame}

\section{The Art of Small Talk}
\subsection{Taking Your Small Talk to a Networking Event}
\begin{frame}
	\frametitle{14.4 The Art of Small Talk}
	Many of us already know that engaging in small talk is intimidating and something most people tend to hate. The problem here is that small talk is a major step in the process of building professional relationships that can give your career a boost, so it is something that cannot be ignored. Here are a few great tips from experts on how you can improve your small talk skills.
\end{frame}
\begin{frame}
	\frametitle{14.4.1 Taking Your Small Talk to a Networking Event}
	Most people feel nervous when entering a room filled with strangers, but it can be much worse for someone who is not naturally sociable or talkative. You need to remember that you are not the only one in the room feeling like that, as most of the attendees there feel the same way!
	
	It is really helpful to know that others are nervous as well. Jim Kokocki, the international president of Toastmasters, once stated, "Walk into any room and take a minute to look around.Grab a drink and go to find someone whos standing awkwardly himself. Say something like, "Hello, my name is Jim. What brings you to this event?" The process can be easier to deal with if you look for other people who are as uncomfortable as you."
\end{frame}
\begin{frame}
	\frametitle{14.4.1 Taking Your Small Talk to a Networking Event}
	Setting tiny and achievable goals can give you something to focus your efforts on, a few experts suggest that. A goal can be as small as meeting a new person at the next coffee break. Once you do this, have a break and drink some water. Your confidence will get a boost by setting small goals that you actually can achieve. Most times youll find that once you begin a conversation, things become much easier and you'll be all right.
\end{frame}
\subsection{Refining Your SmallTalk and Becoming an Expert}
\begin{frame}
	\frametitle{14.4.2 Refining Your SmallTalk and Becoming an Expert}
	\normalsize Along with many other things, there are a few things to remember when it comes to small talk.
	
	Prepare yourself for topics in advance. Dont start thinking about new topics to discuss once uncomfortable silence occurs. It is during that awkward silence that it is the worst time to think of something new. The best strategy is to prepare a few other subjects that you can easily talk about prior to going to any event, as well as conversation ice-breakers like, “Have you been here before?”
	
	The use of open-ended questions should help you not only get a conversation started, but it will also keep a conversation going if you have good follow-up questions.
\end{frame}
\subsection{Taking Control}
\begin{frame}
	\frametitle{14.4.3 Taking Control}
	Most experts agree that taking on the burden of other people's comfort is the key to small talk. This requires you to take control. When you are asked a question, ensure that you have a really good answer ready so that the other person will be more at ease and allow the conversation to continue to flow easier for both of you.
	
	Being an active listener is also important, so dont just stand there nodding your head and occasionally saying single words like "yes", or "really". This activeness lets people know that you are listening to them and are interested in what they have to say, not just hearing their talks and tuning out.
\end{frame}
\subsection{What Is Off Limits?}
\begin{frame}
	\frametitle{14.4.4 What Is Off Limits?}
	It may seem logical to avoid conversations about politics or religion, but it might not always be the case. If you approach these topics in the right way, both can be interesting. Just remember that you need to accept that theres nothing wrong with having a different opinion or being opposed to something. You can always ask for opinions, just avoid telling them that your own view is right. Things can be fine if you simply accept the difference.
	
	Despite that, in a formal setting, you must walk a fine line. There are very few benefits to discuss controversial topics unless you are sure of the group you are dealing with when other options are available.
\end{frame}
\subsection{What Happens if Things Go Wrong?}
\begin{frame}
	\frametitle{14.4.5 What Happens if Things Go Wrong?}
	There are options available to ease any tension brought up when someone discusses a topic that causes things to get heated or causes division in the group. Allow a person to finish what he is saying, then respond with something like "Let's see how it plays out, but that is really interesting", and at last, try to avoid any further stress by moving the conversation to a new topic.
	
	When this happens, make sure that you try to diffuse the situation without being confrontational. One way would be to say, "It's been great to meet you, but I'm going to mingle some more while we still have the chance." This allows you to get out of a conversation without causing any offence.
\end{frame}
\subsection{Keep Your Focus on Others}
\begin{frame}
	\frametitle{14.4.6 Keep Your Focus on Others}
	Constantly talking about yourself is the easiest way to end a conversation, according to many experts. Some experts suggest that talking about yourself for more than four minutes should be avoided as you are overdoing it.
	
	A better suggestion is to focus on the other people involved in the conversation. This will help avoid an awkward situation where were merely focusing on ourselves and aim to make others feel good about themselves by simply showing our interest in them. The result of this is a much more enjoyable and beneficial conversation for all participants.
	
	Using questions to find out more about the other person is another useful technique to maintain a conversation. If the other person mentions that he just returned from a vacation overseas, ask him about it, such as where he went, what the food and weather were like, what he did, etc. This sets you up for answers that allow you to engage more, such as, "Oh, I've never been there." or "I've never tried that sort of food." This will create an easy back and forth conversation that is an example of good small talk.
\end{frame}
\subsection{No One Starts Out as an Expert}
\begin{frame}
	\frametitle{14.4.7 No One Starts Out as an Expert}
	People who are considered masters of small talk and have written books on the subject all started otft as beginners. Everyone is nervous and unsure at the beginning, but learning, understanding, and improving is all the same process. With a bit of guidance and enough practice, anyone can improve his skills at making small talk.
	
	Like any other skills, the focus must be on practice, as the more you do something, the better you improve. Why not purposely seek out a networking event to practice your conversation skills so that you can master the art of small talk?
	
	Set a goal of meeting a new stranger every day and see how you go. You can find a number of apps to help you with this, such as Way of Life, Coach.me, or Balanced. A toastmaster club can give you a more formal method of training. A training program from it can take up to 18 months to finish, but it is quite comprehensive.
\end{frame}

\makelast

\end{document}